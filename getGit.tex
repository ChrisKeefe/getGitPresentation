% This program can be redistributed and/or modified under the terms
% of the GNU Public License, version 3.
%
% Seth Brown, Ph.D.
% sethbrown@drbunsen.org
%
% Compiled with XeLaTeX
% Dependencies:
%   Fontin Sans font (http://www.exljbris.com/fontinsans.html)
%

% TODO: Add citations as bibtex

\documentclass[unknownkeysallowed]{beamer}

\usepackage{graphicx} % graphics
\usepackage{epsfig} % eps graphics
\usepackage{hyperref} % urls
\usepackage{booktabs, caption} % table styling

% suppress navigation bar
\beamertemplatenavigationsymbolsempty

\mode<presentation>
{
  \usetheme{bunsen}
  \setbeamercovered{transparent}
  \setbeamertemplate{items}[circle]
}

% set fonts
\usepackage{fontspec}
\setsansfont{Fontin Sans}
\setbeamerfont{frametitle}{size=\LARGE,series=\bfseries}

% color definitions
\usepackage{color}
\definecolor{uipoppy}{RGB}{225, 64, 5}
\definecolor{uipaleblue}{RGB}{96,123,139}
\definecolor{uiblack}{RGB}{0, 0, 0}

% caption styling
\DeclareCaptionFont{uiblack}{\color{uiblack}}
\DeclareCaptionFont{uipoppy}{\color{uipoppy}}
\captionsetup{labelfont={uipoppy},textfont=uiblack}

% see the macros.tex file for definitions
\include{macros}

% title slide definition
\title{Get Git}
\author{Chris Keefe, Anthony Simard}
\institute{Northern Arizona University \\
Department of Computer Science \\
School of Informatics, Computing, and Cyber Systems \\
}

\date{\today}

%--------------------------------------------------------------------
%                           Introduction
%--------------------------------------------------------------------

\begin{document}

\section{Introduction}
\setbeamertemplate{background}
{\includegraphics[width=\paperwidth,height=\paperheight]{assets/frontpage_bg}}
\setbeamertemplate{footline}[default]

\begin{frame}
\vspace{2cm}
\begin{columns}
\column{2.75in}
  \titlepage
  \vspace{10cm}
\column{2.0in}
\end{columns}
\end{frame}

%-------------------------------------------------------------------
%                          Section 1
%-------------------------------------------------------------------
%
% Set the background for the rest of the slides.
% Insert infoline
\setbeamertemplate{background}
 {\includegraphics[width=\paperwidth,height=\paperheight]{assets/slide_bg}}
\setbeamertemplate{footline}[bunsentheme]

\section{What, Why, and How?}
\begin{frame}
    \frametitle{What is git?}
    \begin{itemize}
        \item{Git is a powerful version control system}
            \begin{itemize}
                \item{Git allows multiple people to easily update the same source}
                \item{Git makes keeping track of changes easy and robust}
            \end{itemize}
        \item{Git allows people around the world to collaborate on open and closed source projects}
    \end{itemize}
    \vspace{1cm} % generate some space between title and content

    % Keeping all of this for now as an example in case we need to do something like this
    %\begin{table}[h]
    %\centering
    %\begin{tabular}{lcccc} \bottomrule[2pt]
    %    Name & Symbol & $A_r$ & M.P. (K) & IE (J) \\ \bottomrule
    %    Helium & He & $4.00$ & $1$ & $3.94e^{-18}$ \\
    %    Carbon & C & $12.01$ & $773$ & $3.94e^{-18}$ \\
    %    Arsenic & As & $74.92$ & $1090$ & $1.48e^{-18}$ \\
    %    Gold & Au & $196.96$ & $1337$ & $1.48e^{-18}$ \\
    %    Cobalt & Co & $58.93$ & $1495$ & $1.26e^{-18}$ \\
    %\bottomrule[2pt]
    %\end{tabular}
    %\caption{Properties of Whoville Elements}
    %\end{table}

    %\vspace{-0.6cm} % compact spacing between table and text

    %\begin{columns}[t]
    %\column{4.5cm}
    %\begin{block}{Trace rare earth metals:}
    %\begin{itemize}
    %    \item{Ytterbium}
    %    \item{Neodymium}
    %    \item{Praseodymium}
    %\end{itemize}
    %\end{block}
    %\column{4.5cm}
    %\begin{block}{Obtaining Neodymium:}
    %    \vspace{0.15cm}
    %    \circled{1}$\textemdash$bastn\"{a}site \\
    %    \circled{2}$\textemdash${monazite} \\
    %\end{block}
    %\end{columns}

\end{frame}

\begin{frame}
    \frametitle{Git is not GitHub}
    \begin{itemize}
        \item{Git is a version control system}
        \item{GitHub is a site cataloging some git repositories}
            \begin{itemize}
                \item{It is free to put open source repos on GitHub}
                \item{You have to pay to catalog closed source repos}
                \item{Not all git repos are catologed on GitHub}
                \item{Git repos absolutely do not need to be on GitHub}
                \item{There are tons of open source repos on GitHub that you can go and check out right now}
           \end{itemize}
    \end{itemize}
    \vspace{1cm} % generate some space between title and content
\end{frame}

\begin{frame}
    \frametitle{Git is not GitHub}
    \begin{itemize}
        \item{Git is a version control system}
        \item{GitHub is a site cataloging some git repositories}
            \begin{itemize}
                \item{It is free to put open source repos on GitHub}
                \item{You have to pay to catalog closed source repos}
                \item{Not all git repos are catologed on GitHub}
                \item{Git repos absolutely do not need to be on GitHub}
                \item{There are tons of open source repos on GitHub that you can go and check out right now}
                \begin{itemize}
                  \item{Flight Rules for Git: https://github.com/k88hudson/git-flight-rules}
                  \item{Visual Studio Code: https://github.com/Microsoft/vscode}
                  \item{Dungeon Crawl Stone Soup: https://github.com/crawl/crawl}
                  \item{Tensorflow: https://github.com/tensorflow/tensorflow}
                  \item{freeCodeCamp: https://github.com/freeCodeCamp/freeCodeCamp}
                  \item{Bootstrap: https://github.com/twbs/bootstrap}
                  \item{TheAlgorithms: https://github.com/TheAlgorithms/Python}
                \end{itemize}
           \end{itemize}
    \end{itemize}
    \vspace{1cm} % generate some space between title and content
\end{frame}

\begin{frame}
    \frametitle{Who uses git?}
    \begin{itemize}
        \item{Everyone uses git}
            \begin{itemize}
                \item{Major corporations}
                \item{Governments}
                \item{Independent software developers}
                \item{Students collaborating on projects}
                \item{Many MANY open source projects}
           \end{itemize}
        \item{Git can easily be used for open or closed source projects making it a diverse and powerful tool}
        \item{We used git and GitHub as version control for this presentation}
    \end{itemize}
    \vspace{1cm} % generate some space between title and content
\end{frame}

\begin{frame}
    \frametitle{Why do I use git?}
    \begin{itemize}
        \item{It's required by my job.}
        \item{Alongside GitHub, git makes showing off easy.}
        \item{It gives me fast, free, highly redundant version control.}
        \item{It makes it easier to work with other people.}
        \item{It makes contributing to open source much easier.}
    \end{itemize}
    \vspace{1cm} % generate some space between title and content
\end{frame}

\section{Getting Started With Git}
\begin{frame}
    \frametitle{Git and Operating Systems}
    \begin{itemize}
        \item{Windows: Install gitbash.}
        \item{Linux: Install through package manager.}
        \item{Mac: Download git-osx-installer and run. }
    \end{itemize}
    \vspace{1cm} % generate some space between title and content
\end{frame}

\begin{frame}
    \frametitle{Our Sample Workflow}
    Simple graphic displaying our workflow with labels for
    \begin{itemize}
        \item{working dir}
        \item{local repo}
        \item{origin repo}
        \item{upstream repo}
    \end{itemize}
    \vspace{1cm} % generate some space between title and content
\end{frame}

\begin{frame}
    \frametitle{The Master Branch is Sacred}
    "There's only one rule: anything in the master branch is always deployable."
    Display same workflow image as above.
    \begin{itemize}
        \item{Project-wide: The product can always ship. You always have a reservoir of clean code to pull from.}
        \item{Locally: Prevents merge conflicts by ensuring you always have a clean copy to branch from.}
        \item{origin repo}
        \item{upstream repo}
    \end{itemize}
    \vspace{1cm} % generate some space between title and content
\end{frame}

\begin{frame}
    \frametitle{Get source code}
    \begin{itemize}
        \item{Note: For this example, we will be working on an existing repo. For your own projects, it's easy to initialize a new repo. Google "git init".}
        \item{Note: during the presentation, define forking, and note that it is a github, not git process.}
        \item{Fork the repo to which you want to contribute - this creates a complete copy of the repository in the cloud, to which you have edit access.}
        \item{Clone your fork - this copies it to your local machine, where you can work with it.  ``` git clone <paste repo url>```}
    \end{itemize}
    \vspace{1cm} % generate some space between title and content
\end{frame}

\begin{frame}
    \frametitle{Explore and make Changes}
    \begin{itemize}
        \item{Navigate into the repo. ```cd <repo-name>``` You'll see:}
            \begin{itemize}
              \item{source files}
              \item{utility files, including a README and license}
              \item{a hidden .git folder, where the magic lives}
            \end{itemize}
        \item{```git status``` what changes have been made in your repo?}
        \item{```git log``` - what does your commit history look like?}
        \item{```git branch``` - what branch have I checked out?}
        \item{```git remote -v``` - what remote copies of the repo does my git know about?}
    \end{itemize}
    \vspace{1cm} % generate some space between title and content
\end{frame}

\begin{frame}
    \frametitle{Make Changes, and Commit Them}
    \begin{itemize}
        \item{```git checkout -b <branchName>``` - this creates and checks out a new branch, keeping your 'master' clean}
        \item{Edit with your preferred editor. Make changes. Save them.}
        \item{Return to the command line, and git status}
        \item{```git add <filenames>``` tells git which files to track in this commit.}
        \item{```git commit -m 'describes the content of the current commit'``` git saves a record of this point in history to your local machine.}
    \end{itemize}
    \vspace{1cm} % generate some space between title and content
\end{frame}

\begin{frame}
    \frametitle{Push to your Fork}
    \begin{itemize}
        \item{```git push <repo-name> <branch-name>``` - pushes your changes to another repository }
        \item{Git is decentralized. Push can send update any repo to which you have edit access.}
        \item{Your changes may now be public. Consider others before rewriting history.}
    \end{itemize}
    \vspace{1cm} % generate some space between title and content
\end{frame}

\begin{frame}
    \frametitle{Open a Pull Request}
    \begin{itemize}
        \item{Pull requests let the owner of the repository incorporate your branch into their codebase}
        \item{Note: A pull request will merge all new commits on the branch.}
        \item{Be sure to follow the contribution guidelines of the repo owner.}
        \item{Expect constructive criticism and requests for code changes.}
    \end{itemize}
    \vspace{1cm} % generate some space between title and content
\end{frame}

\section{Things to Look Forward To!}
\begin{frame}
    \frametitle{Uh-oh!}
    \begin{itemize}
        \item{TREE DIAGRAM}
        \item{You forgot to create a new branch from master}
        \begin{itemize}
            \item{stash your changes, checkout a new branch, stash pop your changes on the new branch}
            \item{If you've already made multiple commits to master, your process will be more complicated.}
        \end{itemize}
        \item{You forgot to update your source before editing}
        \begin{itemize}
            \item{stash or commit your changes, checkout master, pull code from upstream, merge that code with your working branch, stash pop if necessary}
        \end{itemize}
        \item{You pushed the wrong thing to your fork}
        \begin{itemize}
            \item{if anyone else is using that code, make changes as new commits}
            \item{If you're sure you're the only user, it's possible to "force push" changes to the repo}
        \end{itemize}
        \item{You made a horrible, tangled mess of your commit history}
        \begin{itemize}
            \item{Breathe. Use git's fancy log options, or a GUI to visualize our mess.}
            \item{Don't panic. Draw diagrams, and get clear on what needs to happen. Google is your friend.}
            \item{Go for a walk, and come back to it. Git stores all kinds of information on your repo - even "irreversible" damage can usually be fixed with the help of ```git reflog```}
        \end{itemize}
    \end{itemize}
    \vspace{1cm} % generate some space between title and content
\end{frame}

\begin{frame}
    \frametitle{On the bright side...}
    \begin{itemize}
        \item{You'll never lose committed changes.}
        \item{You have fine-grained control over your codebase and its history}
        \item{Sharing your work with collaborators and possible employers is super easy.}
        \item{Editing your code on different machines is super easy}
        \item{You can contribute effectively to all kinds of amazing projects.}
    \end{itemize}
    \vspace{1cm} % generate some space between title and content
\end{frame}

\section{Let's do this!}
\begin{frame}
    \frametitle{Project Overview}
    \begin{itemize}
        \item{Describese the project in steps}
    \end{itemize}
    \vspace{1cm} % generate some space between title and content
\end{frame}

\begin{frame}
    \frametitle{Git Cheat Sheet}
    \begin{itemize}
        \item{TREE DIAGRAM}
    \end{itemize}
    \vspace{1cm} % generate some space between title and content
\end{frame}

\end{document}
