% This program can be redistributed and/or modified under the terms
% of the GNU Public License, version 3.
%
% Seth Brown, Ph.D.
% sethbrown@drbunsen.org
%
% Compiled with XeLaTeX
% Dependencies:
%   Fontin Sans font (http://www.exljbris.com/fontinsans.html)
%

% TODO: Add citations as bibtex

\documentclass[unknownkeysallowed]{beamer}

\usepackage{graphicx} % graphics
\usepackage{epsfig} % eps graphics
\usepackage{hyperref} % urls
\usepackage{booktabs, caption} % table styling

% suppress navigation bar
\beamertemplatenavigationsymbolsempty

\mode<presentation>
{
  \usetheme{bunsen}
  \setbeamercovered{transparent}
  \setbeamertemplate{items}[circle]
}

% set fonts
\usepackage{fontspec}
\setsansfont{Fontin Sans}
\setbeamerfont{frametitle}{size=\LARGE,series=\bfseries}

% color definitions
\usepackage{color}
\definecolor{uipoppy}{RGB}{225, 64, 5}
\definecolor{uipaleblue}{RGB}{96,123,139}
\definecolor{uiblack}{RGB}{0, 0, 0}

% caption styling
\DeclareCaptionFont{uiblack}{\color{uiblack}}
\DeclareCaptionFont{uipoppy}{\color{uipoppy}}
\captionsetup{labelfont={uipoppy},textfont=uiblack}

% see the macros.tex file for definitions
\include{macros}

% title slide definition
% TODO: Change image to something less creepy - complex knot pattern?
\title{Get Git}
\author{Chris Keefe, Anthony Simard, Carter Taylor}
\institute{Northern Arizona University \\
Department of Computer Science \\
School of Informatics, Computing, and Cyber Systems \\
}

\date{\today}

%--------------------------------------------------------------------
%                           Introduction
%--------------------------------------------------------------------

\begin{document}

\section{Introduction}
\setbeamertemplate{background}
{\includegraphics[width=\paperwidth,height=\paperheight]{assets/frontpage_bg}}
\setbeamertemplate{footline}[default]

\begin{frame}
\vspace{2cm}
\begin{columns}
\column{2.75in}
  \titlepage
  \vspace{10cm}
\column{2.0in}
\end{columns}
\end{frame}

%-------------------------------------------------------------------
%                          Section 1
%-------------------------------------------------------------------
%
% Set the background for the rest of the slides.
% Insert infoline
\setbeamertemplate{background}
{\includegraphics[width=\paperwidth,height=\paperheight]{assets/slide_bg}}
\setbeamertemplate{footline}[bunsentheme]

\section{Getting Started With Git}
\begin{frame}
    \frametitle{Installing Git}
    Packages available from: https://git-scm.com/downloads
    \vspace{0.5cm}
    \begin{itemize}
        \item{Windows: Install gitbash.}
        \item{Linux: Install through package manager.}
        \item{Mac: Download git-osx-installer and run. }
    \end{itemize}
    \vspace{0.5cm}
    Note: We will be using Github today. Consider opening a free account now.
    Ditto for Hacktoberfest, if you plan to participate this year.
\end{frame}


\section{What, Why, and How?}
\begin{frame}
    \frametitle{Git is not GitHub}
    \vspace{0.5cm} % generate some space between title and content
    Git is a powerful version control system
            \begin{itemize}
                \item{Git allows multiple people to easily update the same source}
                \item{Git makes keeping track of changes easy and robust}
       		\item{Git allows people around the world to collaborate on open and closed source projects}
            \end{itemize}
    \vspace{0.25cm}
    GitHub is a site hosting some git repositories
            \begin{itemize}
		\item{It is free to host open source repos. Small private repos too.}
                \item{Large closed-source repos cost \$\$\$}
        	\item{GitHub provides communication and project management tools}
	        \begin{itemize}
		    \item{Pull requests}
		    \item{Issue tracking}
		    \item{Project management}
		    \item{Acess permissioning}
		\end{itemize}
    \vspace{0.25cm}
	\item{Git repos absolutely do not need to be on GitHub}
    \end{itemize}
\end{frame}

\begin{frame}
    \frametitle{Some neat repos}
    \vspace{0.75cm}
    Check these out:
    \vspace{0.25cm}
    \begin{itemize}
        \item{Flight Rules for Git: https://github.com/k88hudson/git-flight-rules}
        \item{Visual Studio Code: https://github.com/Microsoft/vscode}
        \item{Dungeon Crawl Stone Soup: https://github.com/crawl/crawl}
        \item{Tensorflow: https://github.com/tensorflow/tensorflow}
        \item{freeCodeCamp: https://github.com/freeCodeCamp/freeCodeCamp}
        \item{Bootstrap: https://github.com/twbs/bootstrap}
        \item{TheAlgorithms: https://github.com/TheAlgorithms/Python}
    \end{itemize}
\end{frame}

\begin{frame}
    \frametitle{Who uses git?}
    \vspace{.5cm}
        \begin{figure}
            \minipage{0.25\textwidth}
                \begin{center}
                    \includegraphics[width = .4\linewidth]{assets/logos/microsoft_logo}
                \end{center}
            \endminipage
            \minipage{0.25\textwidth}
                \begin{center}
                    \includegraphics[width = .4\linewidth]{assets/logos/amazon_logo}
                \end{center}
            \endminipage
            \minipage{0.25\textwidth}
                \begin{center}
                    \includegraphics[width = .4\linewidth]{assets/logos/apple_logo}
                \end{center}
            \endminipage
            \minipage{0.25\textwidth}
                \begin{center}
                    \includegraphics[width = .4\linewidth]{assets/logos/facebook_logo}
                \end{center}
            \endminipage
        \end{figure}
    \begin{itemize}
        \item{Everyone uses git}
            \begin{itemize}
                \item{Major corporations}
                \item{Governments}
                \item{Independent software developers}
                \item{Students collaborating on projects}
                \item{Many MANY open source projects}
           \end{itemize}
        \item{Git can easily be used for open or closed source projects making it a diverse and powerful tool}
        \item{We used git and GitHub as version control for this presentation}
    \end{itemize}
\end{frame}

\begin{frame}
    \frametitle{Why use git?}
    \begin{itemize}
    	\item{Fast, free, safe version control for your own code}
        \vspace{0.25cm}
        \item{Employers use version control}
        \vspace{0.25cm}
        \item{Collaboration is easier with git}
        \vspace{0.25cm}
        \item{Alongside GitHub, Git makes showing off easy}
        \vspace{0.25cm}
        \item{Git skills simplify contributing to open source.}
    \end{itemize}
\end{frame}

\section{Some Vocabulary}
\begin{frame}
	\frametitle{Repository: A fancy bucket}
	\begin{itemize}
	    \item{Repositories can be on your local computer, a friend's machine, or on the cloud.}
        \vspace{0.5cm}
	    \item{Git repositories hold special hidden files that describe your project's history}
        \vspace{0.5cm}
	    \item{You can perform `git` commands from within a repository}
	\end{itemize}
\end{frame}

\begin{frame}
	\frametitle{Commit: the basic unit of change}
	\begin{itemize}
	    \item{A commit is a thematically grouped set of changes}
        \vspace{0.5cm}
	    \item{Commits let us understand and manipulate our development history}
        \vspace{0.5cm}
	    \item{Group and name your commits well, so they're easy to work with.}
	\end{itemize}
\end{frame}

\begin{frame}[fragile]
	\frametitle{Branch: development history}
	A is a commit, with a really bad name.

\begin{verbatim}



A  
o



\end{verbatim}
\end{frame}

\begin{frame}[fragile]
	\frametitle{Branch: development history}
	Your software project grows across time.
\begin{verbatim}



A  
o



\end{verbatim}
\end{frame}

\begin{frame}[fragile]
	\frametitle{Branch: development history}
	    Your software project grows across time.
\begin{verbatim}



A  B
o--o
   


\end{verbatim}
\end{frame}

\begin{frame}[fragile]
	\frametitle{Branch: development history}
	    Your software project grows across time.
\begin{verbatim}



A  B  C
o--o--o
      
     
    
\end{verbatim}
\end{frame}

\begin{frame}[fragile]
	\frametitle{Branch: development history}
	You develop many features on your project's branches.
\begin{verbatim}



A  B  C  
o--o--o--\
          \ 
           \--o
              D
\end{verbatim}
\end{frame}

\begin{frame}[fragile]
	\frametitle{Branch: development history}
	You develop many features on your projects branches.
\begin{verbatim}

             
              
A  B  C     E
o--o--o--\--o
          \ 
           \--o
              D  
\end{verbatim}
\end{frame}

\begin{frame}[fragile]
	\frametitle{Branch: development history}
	You develop many features on your projects branches.
\begin{verbatim}
                   F
                /--o
               /
A  B  C     E /
o--o--o--\--o/
          \ 
           \--o
              D  
\end{verbatim}
\end{frame}

\begin{frame}[fragile]
	\frametitle{Branch: development history}
	You develop many features on your projects branches.
\begin{verbatim}
                   F
                /--o
               /
A  B  C     E /  G
o--o--o--\--o/---o
          \ 
           \--o
              D  
\end{verbatim}
\end{frame}

\begin{frame}[fragile]
	\frametitle{Branch: development history}
	You develop many features on your projects branches.
\begin{verbatim}
                   F
                /--o
               /
A  B  C     E /  G
o--o--o--\--o/---o
          \ 
           \--o--o--\
              D  H   \
                      \--o
                        DUMB
\end{verbatim}
\end{frame}

\begin{frame}[fragile]
	\frametitle{Branch: development history}
	You develop many features on your projects branches.
\begin{verbatim}
                   F
                /--o
               /
A  B  C     E /  G
o--o--o--\--o/---o
          \            
           \--o--o--\--o
              D  H   \ I
                      \--o
                        DUMB
\end{verbatim}
\end{frame}

\begin{frame}[fragile]
	\frametitle{Branch: development history}
	When harvest time comes, you pick the good ones and leave the bad.
\begin{verbatim}
                   F
                /--o
               /
A  B  C     E /  G
o--o--o--\--o/---o
          \            
           \--o--o--\--o
              D  H   \ I
                      \--o
                        DUMB
\end{verbatim}
\end{frame}

\begin{frame}[fragile]
	\frametitle{Branch: development history}
	When harvest time comes, you pick the good ones and leave the bad.
\begin{verbatim}
                   F
                /--o
               /
A  B  C     E /  G  F
o--o--o--\--o/---o--o
          \            
           \--o--o--\--o
              D  H   \ I
                      \--o
                        DUMB
\end{verbatim}
\end{frame}

\begin{frame}[fragile]
	\frametitle{Branch: development history}
	When harvest time comes, you pick the good ones and leave the bad.
\begin{verbatim}
                   F
                /--o
               /
A  B  C     E /  G  F      D  H  I  J(merged)
o--o--o--\--o/---o--o-----/o--o--o--o
          \              /
           \--o--o--\--o/
              D  H   \ I
                      \--o
                        DUMB
\end{verbatim}
\end{frame}

\begin{frame}[fragile]
	\frametitle{Branch: development history}
	It's possible to clean up unused branches at any time.
\begin{verbatim}
                   F
                /--o
               /
A  B  C     E /  G  F      D  H  I  J(merged)
o--o--o--\--o/---o--o-----/o--o--o--o
          \              /
           \--o--o--\--o/
              D  H   \ I
                      \--o
                        DUMB
\end{verbatim}
\end{frame}

\begin{frame}[fragile]
	\frametitle{Branch: development history}
	It's possible to clean up unused branches at any time.
\begin{verbatim}
                   F
                /--o
               /
A  B  C     E /  G  F      D  H  I  J(merged)
o--o--o--\--o/---o--o-----/o--o--o--o
          \              /
           \--o--o-----o/
              D  H    I
                      
\end{verbatim}
\end{frame}

\begin{frame}[fragile]
	\frametitle{Branch: development history}
	It's possible to clean up unused branches at any time.
\begin{verbatim}
                   
                
               
A  B  C     E    G  F      D  H  I  J(merged)
o--o--o--\--o----o--o-----/o--o--o--o
          \              /
           \--o--o-----o/
              D  H    I
                      
\end{verbatim}
\end{frame}

\begin{frame}[fragile]
	\frametitle{Branch: development history}
	It's possible to clean up unused branches at any time.
\begin{verbatim}
                   
                
               
A  B  C     E    G  F      D  H  I  J(merged)
o--o--o-----o----o--o------o--o--o--o
          
          
          
                      
\end{verbatim}
\end{frame}

\begin{frame}
    \frametitle{The Master Branch is Sacred}
    "There's only one rule: anything in the master branch is always deployable."
    \vspace{0.25cm}
    \begin{itemize}
        \item{Project-wide: The product can always ship. You always have a reservoir of clean code to pull from.}
        \item{Locally: Prevents merge conflicts by ensuring you always have a clean copy to branch from.}
        \item{origin repo vs upstream repo}
    \end{itemize}
\end{frame}

% My super roughed up workflow is still there for now just so that slide has something
\setbeamertemplate{background}
{\includegraphics[width=\paperwidth,height=\paperheight]{assets/workflow_bg}}
\begin{frame}
    \frametitle{Our Sample Workflow}
        1. Pull from Upstream Master to \\
        Local Master \linebreak\linebreak
        2. Branch to Working Directory \linebreak\linebreak
        3. Make and test changes \linebreak\linebreak
        4. Commit to Local Dev \linebreak\linebreak
        5. Push to Origin Dev \linebreak\linebreak
        6. Open pull request to Upstream \\
        Master
\end{frame}
\setbeamertemplate{background}
{\includegraphics[width=\paperwidth,height=\paperheight]{assets/slide_bg}}

\begin{frame}[fragile]
    \frametitle{Get source code}
Today, we will be updating an existing Github repo. For your own projects, it's easy to initialize a new repo. Google "git init".
You will need:
    \begin{itemize}
        \item{Buddies to work with}
        \item{A computer with Git, an editor, and a Github account}
        \item{A "Fork" of the repo to which you want to contribute - a fork is a complete copy of the repository, to which you have edit access. Note: this is a Github concept}
        \item{A "Clone" of your fork - this copies it to your local machine, where you can work with it.  \verb/git clone <paste repo url>/}
    \end{itemize}
\end{frame}

\begin{frame}
    \frametitle{Explore your repo}
    \begin{itemize}
        \item{Navigate into the cloned repo. ```cd <repo-name>``` You'll see:}
            \begin{itemize}
              \item{source files}
              \item{utility files, including a README and license}
              \item{a hidden .git folder, where the magic lives}
            \end{itemize}
        \item{```git status``` what changes have been made in your repo?}
        \item{```git log``` - what does your commit history look like?}
        \item{```git branch``` - what branch have I checked out?}
        \item{```git remote -v``` - what remote copies of the repo does my git know about?}
    \end{itemize}
\end{frame}

\begin{frame}
    \frametitle{Make Changes, and Commit Them}
    \begin{itemize}
        \item{```git checkout -b <branchName>``` - this creates and checks out a new branch, keeping your 'master' clean}
        \item{Edit with your preferred editor. Make changes. Save them.}
        \item{Return to the command line, and ```git status```}
        \item{```git add <filenames>``` tells git which files to track in this commit.}
        \item{```git commit -m 'describes the content of the current commit'``` git saves a record of this point in history to your local machine.}
    \end{itemize}
\end{frame}

\begin{frame}
    \frametitle{Push to your Fork}
    \begin{itemize}
        \item{```git push <repo-name> <branch-name>``` - pushes your changes to another repository }
        \item{Git is decentralized. Push can update any repo to which you have edit access.}
        \item{Your changes may now be public. Consider others before rewriting history.}
    \end{itemize}
\end{frame}

\begin{frame}
    \frametitle{Open a Pull Request}
    \begin{itemize}
        \item{Pull requests ask the owner of the repository to merge your branch into their codebase}
        \item{A pull request will merge all commits on the branch. Make a new branch if you want to develop a new PR.}
        \item{Be sure to follow the contribution guidelines of the repo owner.}
        \item{Expect constructive criticism and requests for code changes.}
    \end{itemize}
\end{frame}

\section{Things to Look Forward To!}
\begin{frame}
    \frametitle{Uh-oh!}
    \vspace{0.25cm}
    There are many fun ways to break things with Git. Let's get started!
    \begin{itemize}
        \item{You forgot to create a new branch from master}
        \begin{itemize}
            \item{`stash` changes, `checkout` a branch, `stash pop` on the new branch}
            \item{If you've made multiple commits to `master`, try `rebase`}
        \end{itemize}
        \item{You forgot to update your source before editing}
        \begin{itemize}
            \item{`stash` or `commit` your changes, 'checkout master', 'pull' code from 'upstream', 'merge' that code with your working branch, 'stash pop' if necessary}
        \end{itemize}
        \item{You pushed the wrong thing to your fork}
        \begin{itemize}
            \item{if anyone else is using that code, make changes as new commits}
            \item{If you're the only user, you might "force push" changes to the repo}
        \end{itemize}
        \item{You made a horrible, tangled mess of your commit history}
        \begin{itemize}
            \item{Breathe. Use git's fancy `log` options or a GUI to display your mess.}
            \item{Don't panic. Draw diagrams. Make a plan. Google is your friend.}
            \item{Go walking. "Irreversible" damage can usually be fixed by `git reflog`}
        \end{itemize}
    \end{itemize}
\end{frame}

\begin{frame}
    \frametitle{On the bright side...}
    \begin{itemize}
        \item{You'll never lose committed changes.}
        \item{You have fine-grained control over your codebase and its history}
        \item{Sharing your work with collaborators and possible employers is super easy.}
        \item{Editing your code on different machines is super easy}
        \item{You can contribute effectively to all kinds of amazing projects.}
    \end{itemize}
\end{frame}

\section{Let's do this!}
\begin{frame}
    \frametitle{Project Overview}
    What's next?
    \begin{itemize}
        \item{Work closely with another group to edit existing pages and avoid mergeconflict.}
        \item{Create merge conflicts by editing existing pages without coordination.}
        \item{Feeling bold? Contribute to this presentation instead of the test repo.}
        \item{Super pro? Find an open source project, and contribute to that.}
    \end{itemize}
\end{frame}

\begin{frame}
\frametitle{Git Cheat Sheet}
\begin{columns}
    \column{2.75in}
        \begin{itemize}
            \item{```git init```}
            \item{```git clone <repo url>```}
            \item{```git fetch <remote alias>```}
            \item{```git pull <remote alias> <branch>```}
            \item{```git checkout <branch name>```}
            \item{```git checkout -b <branch name>```}
            \item{```git push <remote alias> <branch>```}

        \end{itemize}
    \column{2.0in}
        \begin{itemize}
            \item{```git status```}
            \item{```git git log```}
            \item{```git branch```}
            \item{```git remote -v```}
            \item{```git add <filename>```}
            \item{```git commit -m 'message'```}
        \end{itemize}
    \end{columns}
\end{frame}

\end{document}
