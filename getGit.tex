% This program can be redistributed and/or modified under the terms
% of the GNU Public License, version 3.
%
% Seth Brown, Ph.D.
% sethbrown@drbunsen.org
%
% Compiled with XeLaTeX
% Dependencies:
%   Fontin Sans font (http://www.exljbris.com/fontinsans.html)
%

% TODO: Add citations as bibtex

\documentclass[unknownkeysallowed]{beamer}

\usepackage{graphicx} % graphics
\usepackage{epsfig} % eps graphics
\usepackage{hyperref} % urls
\usepackage{booktabs, caption} % table styling

% suppress navigation bar
\beamertemplatenavigationsymbolsempty

\mode<presentation>
{
  \usetheme{bunsen}
  \setbeamercovered{transparent}
  \setbeamertemplate{items}[circle]
}

% configure hyperlink styling
\hypersetup{
  colorlinks   = true, %Removes boxes and colors links
  urlcolor     = blue,
  linkcolor    = blue,
  citecolor   = red
}


% set fonts
\usepackage{fontspec}
\setsansfont{Fontin Sans}
\setbeamerfont{frametitle}{size=\LARGE,series=\bfseries}

% color definitions
\usepackage{color}
\definecolor{uipoppy}{RGB}{225, 64, 5}
\definecolor{uipaleblue}{RGB}{96,123,139}
\definecolor{uiblack}{RGB}{0, 0, 0}

% caption styling
\DeclareCaptionFont{uiblack}{\color{uiblack}}
\DeclareCaptionFont{uipoppy}{\color{uipoppy}}
\captionsetup{labelfont={uipoppy},textfont=uiblack}

% see the macros.tex file for definitions
\include{macros}

% title slide definition
% TODO: Change image to something less creepy - complex knot pattern?
\title{Get Git}
\author{Chris Keefe, Anthony Simard, Carter Taylor}
\institute{Northern Arizona University \\
Department of Computer Science \\
School of Informatics, Computing, and Cyber Systems \\
}

\date{\today}

%--------------------------------------------------------------------
%                           Introduction
%--------------------------------------------------------------------

\begin{document}

\section{Introduction}
\setbeamertemplate{background}
{\includegraphics[width=\paperwidth,height=\paperheight]{assets/frontpage_bg}}
\setbeamertemplate{footline}[default]

\begin{frame}
\vspace{2cm}
\begin{columns}
\column{2.75in}
  \titlepage
  \vspace{10cm}
\column{2.0in}
\end{columns}
\end{frame}

%-------------------------------------------------------------------
%                          Section 1
%-------------------------------------------------------------------
%
% Set the background for the rest of the slides.
% Insert infoline
\setbeamertemplate{background}
{\includegraphics[width=\paperwidth,height=\paperheight]{assets/slide_bg}}
\setbeamertemplate{footline}[bunsentheme]


\section{What, Why, and How?}
\begin{frame}
    \frametitle{Git is not GitHub}
    \vspace{0.5cm} % generate some space between title and content
    Git is a powerful version control system
            \begin{itemize}
                \item{Git allows multiple people to easily update the same source}
                \item{Git makes keeping track of changes easy and robust}
       		\item{Git allows people around the world to collaborate on open and closed source projects}
            \end{itemize}
    \vspace{0.25cm}
    GitHub hosts some git repositories, and provides communication and project management tools
        \begin{itemize}
	    \item{Pull requests}
	    \item{Issue tracking}
	    \item{Project management}
	    \item{Acess permissioning}
    \end{itemize}
    \vspace{0.25cm}
	Git repos absolutely do not need to be on GitHub
\end{frame}

\begin{frame}
    \frametitle{Who uses git?}
    \vspace{.5cm}
        \begin{figure}
            \minipage{0.25\textwidth}
                \begin{center}
                    \includegraphics[width = .4\linewidth]{assets/logos/microsoft_logo}
                \end{center}
            \endminipage
            \minipage{0.25\textwidth}
                \begin{center}
                    \includegraphics[width = .4\linewidth]{assets/logos/amazon_logo}
                \end{center}
            \endminipage
            \minipage{0.25\textwidth}
                \begin{center}
                    \includegraphics[width = .4\linewidth]{assets/logos/apple_logo}
                \end{center}
            \endminipage
            \minipage{0.25\textwidth}
                \begin{center}
                    \includegraphics[width = .4\linewidth]{assets/logos/facebook_logo}
                \end{center}
            \endminipage
        \end{figure}
    \begin{itemize}
        \item{Everyone uses git}
            \begin{itemize}
                \item{Major corporations}
                \item{Governments}
                \item{Independent software developers}
                \item{Students collaborating on projects}
                \item{Many MANY open source projects}
           \end{itemize}
   \item{\href{https://github.com/Microsoft/vscode}{Visual Studio Code} \hfill \href{https://github.com/tensorflow/tensorflow}{Tensorflow} \hfill \href{https://github.com/twbs/bootstrap}{Bootstrap}}
        \item{We used git and GitHub as version control for this presentation}
    \end{itemize}
\end{frame}

\begin{frame}
	\frametitle{Why version control?}
	\begin{itemize}
		\item{Safely manage versions and branches of documents}
		\item{Integrate changes from multiple systems/collaborators/feature branches}
		\item{Document redundancy and recovery}
	\end{itemize}
\end{frame}


\section{Some Vocabulary}
\begin{frame}
	\frametitle{Repository: A fancy bucket}
	\begin{itemize}
	    \item{Repositories can be on your local computer, a friend's machine, or on the cloud.}
        \vspace{0.5cm}
	    \item{Git repositories hold special hidden files that describe your project's history}
        \vspace{0.5cm}
	    \item{You can perform `git` commands from within a repository}
	\end{itemize}
\end{frame}

\begin{frame}
	\frametitle{Commit: the basic unit of change}
	\begin{itemize}
	    \item{A commit is a thematically grouped set of changes}
        \vspace{0.5cm}
	    \item{Commits let us understand and manipulate our development history}
        \vspace{0.5cm}
	    \item{Group and name your commits well, so they're easy to work with.}
	\end{itemize}
\end{frame}

\begin{frame}[fragile]
	\frametitle{Branch: development history}
	A is a commit, with a really bad name.

\begin{verbatim}



A  
o



\end{verbatim}
\end{frame}

\begin{frame}[fragile]
	\frametitle{Branch: development history}
	Your software project grows across time.
\begin{verbatim}



A  
o



\end{verbatim}
\end{frame}

\begin{frame}[fragile]
	\frametitle{Branch: development history}
	    Your software project grows across time.
\begin{verbatim}



A  B
o--o
   


\end{verbatim}
\end{frame}

\begin{frame}[fragile]
	\frametitle{Branch: development history}
	    Your software project grows across time.
\begin{verbatim}



A  B  C
o--o--o
      
     
    
\end{verbatim}
\end{frame}

\begin{frame}[fragile]
	\frametitle{Branch: development history}
	You develop many features on your project's branches.
\begin{verbatim}



A  B  C  
o--o--o--\
          \ 
           \--o
              D
\end{verbatim}
\end{frame}

\begin{frame}[fragile]
	\frametitle{Branch: development history}
	You develop many features on your projects branches.
\begin{verbatim}

             
              
A  B  C     E
o--o--o--\--o
          \ 
           \--o
              D  
\end{verbatim}
\end{frame}

\begin{frame}[fragile]
	\frametitle{Branch: development history}
	You develop many features on your projects branches.
\begin{verbatim}
                   F
                /--o
               /
A  B  C     E /
o--o--o--\--o/
          \ 
           \--o
              D  
\end{verbatim}
\end{frame}

\begin{frame}[fragile]
	\frametitle{Branch: development history}
	You develop many features on your projects branches.
\begin{verbatim}
                   F
                /--o
               /
A  B  C     E /  G
o--o--o--\--o/---o
          \ 
           \--o
              D  
\end{verbatim}
\end{frame}

\begin{frame}[fragile]
	\frametitle{Branch: development history}
	You develop many features on your projects branches.
\begin{verbatim}
                   F
                /--o
               /
A  B  C     E /  G
o--o--o--\--o/---o
          \ 
           \--o--o--\
              D  H   \
                      \--o
                        DUMB
\end{verbatim}
\end{frame}

\begin{frame}[fragile]
	\frametitle{Branch: development history}
	You develop many features on your projects branches.
\begin{verbatim}
                   F
                /--o
               /
A  B  C     E /  G
o--o--o--\--o/---o
          \            
           \--o--o--\--o
              D  H   \ I
                      \--o
                        DUMB
\end{verbatim}
\end{frame}

\begin{frame}[fragile]
	\frametitle{Branch: development history}
	When harvest time comes, you pick the good ones and leave the bad.
\begin{verbatim}
                   F
                /--o
               /
A  B  C     E /  G
o--o--o--\--o/---o
          \            
           \--o--o--\--o
              D  H   \ I
                      \--o
                        DUMB
\end{verbatim}
\end{frame}

\begin{frame}[fragile]
	\frametitle{Branch: development history}
	When harvest time comes, you pick the good ones and leave the bad.
\begin{verbatim}
                   F
                /--o
               /
A  B  C     E /  G  F
o--o--o--\--o/---o--o
          \            
           \--o--o--\--o
              D  H   \ I
                      \--o
                        DUMB
\end{verbatim}
\end{frame}

\begin{frame}[fragile]
	\frametitle{Branch: development history}
	When harvest time comes, you pick the good ones and leave the bad.
\begin{verbatim}
                   F
                /--o
               /
A  B  C     E /  G  F      D  H  I  J(merged)
o--o--o--\--o/---o--o-----/o--o--o--o
          \              /
           \--o--o--\--o/
              D  H   \ I
                      \--o
                        DUMB
\end{verbatim}
\end{frame}

\begin{frame}[fragile]
	\frametitle{Branch: development history}
	It's possible to clean up unused branches at any time.
\begin{verbatim}
                   F
                /--o
               /
A  B  C     E /  G  F      D  H  I  J(merged)
o--o--o--\--o/---o--o-----/o--o--o--o
          \              /
           \--o--o--\--o/
              D  H   \ I
                      \--o
                        DUMB
\end{verbatim}
\end{frame}

\begin{frame}[fragile]
	\frametitle{Branch: development history}
	It's possible to clean up unused branches at any time.
\begin{verbatim}
                   F
                /--o
               /
A  B  C     E /  G  F      D  H  I  J(merged)
o--o--o--\--o/---o--o-----/o--o--o--o
          \              /
           \--o--o-----o/
              D  H    I
                      
\end{verbatim}
\end{frame}

\begin{frame}[fragile]
	\frametitle{Branch: development history}
	It's possible to clean up unused branches at any time.
\begin{verbatim}
                   
                
               
A  B  C     E    G  F      D  H  I  J(merged)
o--o--o--\--o----o--o-----/o--o--o--o
          \              /
           \--o--o-----o/
              D  H    I
                      
\end{verbatim}
\end{frame}

\begin{frame}[fragile]
	\frametitle{Branch: development history}
	It's possible to clean up unused branches at any time.
\begin{verbatim}
                   
                
               
A  B  C     E    G  F      D  H  I  J(merged)
o--o--o-----o----o--o------o--o--o--o
          
          
          
                      
\end{verbatim}
\end{frame}

\section{Git workflows}
\begin{frame}
    \frametitle{The Master Branch is Sacred}
    "There's only one rule: anything in the master branch is always deployable."
    \vspace{0.25cm}
    \begin{itemize}
        \item{Project-wide: The product can always ship. You always have a reservoir of clean code to pull from.}
        \item{Locally: Prevents merge conflicts by ensuring you always have a clean copy to branch from.}
        \item{origin repo vs upstream repo}
    \end{itemize}
\end{frame}


\begin{frame}
    \vspace{1.2cm}
	\begin{columns}
		\column{2.5in}
    \frametitle{Our Sample Workflow}
	\column{2.25in}
	\begin{center}
	\includegraphics[width = .9\linewidth]{assets/gitflow1}
	\end{center}
	\end{columns}
    \vspace{1cm}
\end{frame}

\begin{frame}
    \vspace{1cm}
	\begin{columns}
		\column{2.5in}
    \frametitle{Our Sample Workflow}
        1. Pull from Upstream Master to \\
        Local Master \linebreak\linebreak
	\linebreak\linebreak 
	\linebreak\linebreak
	\linebreak\linebreak
	\linebreak\linebreak
	\linebreak\linebreak
	\column{2.25in}
	\begin{center}
	\includegraphics[width = .9\linewidth]{assets/gitflow2}
	\end{center}
	\end{columns}
    \vspace{1cm}
\end{frame}

\begin{frame}
    \vspace{1cm}
	\begin{columns}
		\column{2.5in}
    \frametitle{Our Sample Workflow}
        1. Pull from Upstream Master to \\
        Local Master \linebreak\linebreak
	\linebreak\linebreak 
	\linebreak\linebreak
	\linebreak\linebreak
	\linebreak\linebreak
	\linebreak\linebreak
	\column{2.25in}
	\begin{center}
	\includegraphics[width = .9\linewidth]{assets/gitflow3}
	\end{center}
	\end{columns}
    \vspace{1cm}
\end{frame}

\begin{frame}
    \vspace{1cm}
	\begin{columns}
		\column{2.5in}
    \frametitle{Our Sample Workflow}
        1. Pull from Upstream Master to \\
        Local Master \linebreak\linebreak
        2. Branch to Working Directory \linebreak\linebreak
	\linebreak\linebreak
	\linebreak\linebreak
	\linebreak\linebreak
	\linebreak\linebreak
	\column{2.25in}
	\begin{center}
	\includegraphics[width = .9\linewidth]{assets/gitflow4}
	\end{center}
	\end{columns}
    \vspace{1cm}
\end{frame}

\begin{frame}
    \vspace{1cm}
	\begin{columns}
		\column{2.5in}
    \frametitle{Our Sample Workflow}
        1. Pull from Upstream Master to \\
        Local Master \linebreak\linebreak
        2. Branch to Working Directory \linebreak\linebreak
        3. Make and test changes \linebreak\linebreak
	\linebreak\linebreak
	\linebreak\linebreak
	\linebreak\linebreak
	\column{2.25in}
	\begin{center}
	\includegraphics[width = .9\linewidth]{assets/gitflow4}
	\end{center}
	\end{columns}
    \vspace{1cm}
\end{frame}

\begin{frame}
    \vspace{1cm}
	\begin{columns}
		\column{2.5in}
    \frametitle{Our Sample Workflow}
        1. Pull from Upstream Master to \\
        Local Master \linebreak\linebreak
        2. Branch to Working Directory \linebreak\linebreak
        3. Make and test changes \linebreak\linebreak
        4. Commit to Local Dev \linebreak\linebreak
	\linebreak\linebreak
	\linebreak\linebreak
	\column{2.25in}
	\begin{center}
	\includegraphics[width = .9\linewidth]{assets/gitflow4}
	\end{center}
	\end{columns}
    \vspace{1cm}
\end{frame}

\begin{frame}
    \vspace{1cm}
	\begin{columns}
		\column{2.5in}
    \frametitle{Our Sample Workflow}
        1. Pull from Upstream Master to \\
        Local Master \linebreak\linebreak
        2. Branch to Working Directory \linebreak\linebreak
        3. Make and test changes \linebreak\linebreak
        4. Commit to Local Dev \linebreak\linebreak
        5. Push to Origin Dev \linebreak\linebreak
	\linebreak\linebreak
	\column{2.25in}
	\begin{center}
	\includegraphics[width = .9\linewidth]{assets/gitflow5}
	\end{center}
	\end{columns}
    \vspace{1cm}
\end{frame}

\begin{frame}
    \vspace{1.01cm}
	\begin{columns}
		\column{2.5in}
    \frametitle{Our Sample Workflow}
        1. Pull from Upstream Master to \\
        Local Master \linebreak\linebreak
        2. Branch to Working Directory \linebreak\linebreak
        3. Make and test changes \linebreak\linebreak
        4. Commit to Local Dev \linebreak\linebreak
        5. Push to Origin Dev \linebreak\linebreak
        6. Open pull request to Upstream \\
        Master
	\column{2.25in}
	\begin{center}
	\includegraphics[width = .95\linewidth]{assets/gitflow6}
	\end{center}
	\end{columns}
    \vspace{1cm}
\end{frame}

\begin{frame}
    \vspace{.25cm}
    \frametitle{In An Alternate Reality}
	\begin{center}
	\includegraphics[width = .8\linewidth]{assets/thenextlevel}
	\end{center}
    \vspace{.25cm}
\end{frame}

\setbeamertemplate{background}
{\includegraphics[width=\paperwidth,height=\paperheight]{assets/slide_bg}}

\begin{frame}
    \frametitle{So... why git?}
    \begin{itemize}
        \item{You'll never lose committed changes.}
        \item{You have fine-grained control over your codebase and its history}
        \item{Sharing your work with collaborators and possible employers is super easy.}
        \item{Editing your code on different machines is super easy}
        \item{You can contribute effectively to all kinds of amazing projects.}
    \end{itemize}
\end{frame}


\section{Resources}
\begin{frame}
    \frametitle{Install Git}
    Packages available from: \url{https://git-scm.com/downloads}
    \vspace{0.5cm}
    \begin{itemize}
        \item{Windows: Install gitbash.}
        \item{Linux: Install through package manager.}
        \item{Mac: Download git-osx-installer and run. }
    \end{itemize}
    \vspace{0.5cm}
    Note: We will be using Github today, but there are many similar options. 
	Consider opening a free account now.
\end{frame}

\begin{frame}
\frametitle{Git Cheat Sheet}
\begin{columns}
    \column{2.75in}
        \begin{itemize}
            \item{`git init`}
            \item{`git clone <repo url>`}
            \item{`git fetch <remote alias>`}
            \item{`git pull <remote alias> <branch>`}
            \item{`git checkout <branch name>`}
            \item{`git checkout -b <branch name>`}
            \item{`git push <remote alias> <branch>`}

        \end{itemize}
    \column{2.0in}
        \begin{itemize}
            \item{`git status`}
            \item{`git git log`}
            \item{`git branch`}
            \item{`git remote -v`}
            \item{`git add <filename>`}
            \item{`git commit -m 'message'`}
        \end{itemize}
    \end{columns}
	\vspace{.5cm}
	github.com/ChrisKeefe/getGitPresentation/tree/mini-git
\end{frame}

\end{document}
